\chapter{Zespołowe przedsięwzięcie}

\begin{itemize}
\item Zespołowe przedsięwzięcie inżynierskie oznaczać będzie projekt, działanie podjęte w realizacji postawionego celu, realizowane zespołowo.
\item Projekt jest odpowiedzią na problem/potrzebę, w określonej przestrzeni życia.
\end{itemize}

\section{Członkowie zespołu z określeniem funkcji}
\begin{description}
\item[1] Piotr Jabloński - programista Java
\item[2] Mirosława Pelc - programista Java
\item[3] Mariusz Lorek - kierownik zespołu, testowanie, przygotowanie dokumentacji
\end{description}

\section{Uzasadnienie potrzeby realizacji projektu}
Potrzebny jest program który wspomoże zorganizowanie turnieju szachowego w którym może wziąść udział dowolna, nieznana wcześniej liczba zawodników. Czas trwania turnieju jest ograniczony przez organizatora. Turniej szachowy jest organizowany cyklicznie, dlatego stworzenie programu wspomagającego jego obsługę znacznie ułatwi przeprowadzanie kolejnych edycji.


\section{Cele projektu}
\begin{enumerate}
\item Stworzenie programu wspomagającego organizację turnieju szachowego.
Napisany program ma pozwolić na sprawne przeprowadzenie turnieju szachowego i wyłonienie zwycięzcy turnieju i/lub zawodników którzy zajeli kolejne miejsca w turnieju.
\item Przygotowanie instrukcji obsługi programu/aplikacji dla użytkownika końcowego
\end{enumerate}
 


\section{Zakres projektu}
\begin{enumerate}
	\item Stworzenie programu do wspomagania organiacji turnieju szachowego według wytycznych zleceniodawcy
	\item Stworzenie dokumentacji opisującej postępy prac nad tworzonym projektem z podziałem na czynności które ma wykonywać każdy z członków zespołu
\end{enumerate}
Gotowy program ma pozwalać m.in na:
\begin{itemize}
	\item Przeprowadzenie turnieju szachowego w systemie kołowym (każdy z każdym) z podziałem na grupy
	\item Zarzadzanie turniejami w bazie danych 
	\item Zgromadzenie podstawowych danych o zawodnikach, jakim są: (Imię, Nazwisko, Wiek, Kategoria szachowa) 
	\item Dodawanie, usuwanie i edycja zawodników, 
	\item Podział zawodników na grupy w ze względu na przedział wiekowy, kategorie szachową lub manualnie. 
	\item Ustalenie liczby grup oraz szachownic przed rozpoczęciem nowego turnieju. 
	\item Ustalanie uczestników każdego meczu - kolor pionków (biały, czarny) przydzielany do zawodników przed każdym spotkaniem 
	\item Punktowanie rozegranych spotkań
\end{itemize}


\section{Grupy docelowe}
Program przeznaczony dla organizatorów turniejów szachowych.  


\section{Struktura podziału prac (zadań) - WBS}

Program wspomagający przeprowadzenie turneju szachowego
\begin{enumerate}
\item Zebranie informacji na temat sposobu przprowadzania turnieju szachowego od zleceniodawcy
\begin{enumerate}
	\item Wybranie systemu według którego będzie przeprowadzany turniej, wybór najoptymalniejszego rozwiązania
	\item Przygotowanie regulaminu turnieju.
\end{enumerate}
\item Projekt programu
\begin{enumerate}
\item Określenie jakie elementy muszą się znaleść w programie
\item Szablon programu
\item Wybór narzędzi/aplikacji służących do napisania programu
\item Rozdzielenie zadań dla programistów
\end{enumerate}
\item Tworzenie programu/aplikacji
\begin{itemize}
\item Opracowanie narzędzi bazodanowych przechowujących informacje dotyczące turniejów
\item Przygotowanie elementów środowiska graficznego
\item Integracja narzędzi bazodanowych z elementami środowiska graficznego

\item Wstępna wersja programu
\item Testowanie
\begin{enumerate}
	\item Weryfikacja - "Czy budujemy prawidłowo produkt", dynamiczna i statyczna
	\item Walidacja - "Czy budujemy prawidłowy produkt"
	\item Testy
	\begin{itemize}
		\item Testy jednostkowe
		\item Testy integracyjne
		\item Testy systemowe
		\item Testy użyteczności
		\item Testy akceptacyjne (przeprowadzane przez zleceniodawce projektu
	)
	Testy mają za zadanie sprawdzenie każdego komponentu niezależnie
	\end{itemize}
\end{enumerate}
\item Eliminacja znalezionych błędów
\item Dodawanie kolejnych funkcji do programu

\end{itemize}
\item Końcowa wersja programu

\end{enumerate}

\section{Regulamin turnieju}
\begin{enumerate}
\item Obowiązują przepisy gry międzynarodowej federacji szachowej (fide).
\item Każdy z zawodników powinien się kierować zasadami fair play.
\item Zasady rozgrywki:
\begin{enumerate}
\item Turniej rozgrywany jest w dwóch fazach: grupowej i finałowej.\\
\item Tworzona jest lista startowa według przyjętych przez prowadzącego turniej kryteriów, domyślnie:
\begin{enumerate}
\item kategoria szachowa 
\item wiek
\item alfabetycznie
\end{enumerate}
\item Ilość grup jest zależna od liczby uczestników i ustala ją prowadzący.
\item Lista startowa dzielona jest na liczbę części równą liczbie grup
\item Następnie zawodnicy z każdej części są losowo rozmieszczani w grupach
\item W fazie grupowej prowadzone są rozgrywki, gdzie każdy gra z każdym w danej grupie.
\item W fazie finałowej zawodnicy wyłonieni z grup (liczbę osób wychodzących z grup ustala prowadzący) grają między sobą.
\item Turniej trwa maksymalnie 2.5 godziny z tego:
\begin{itemize}
\item 10 minut na przyjmowanie zgłoszeń(rejestrację),
\item 10 minut na losowanie spotkań,
\item Na każdą rozgrywaną partię przypada 10 minut. Każdy zawodnik ma 5 minut na wykonanie swoich posunięć.
\end{itemize}
\item Zawodnicy mają do dyspozycji zegar analogowy lub cyfrowy z 2 tarczami lub wyświetlaczami umożliwiający odmierzanie czasu rozgrywki dla każdego z zawodników osobno
\item Za zajecie ustalonych przez prowadzącego miejsc w turnieju zawodnicy otrzymują nagrody przewidziane przez organizatora.
\item Jeśli prowadzący ustali, uczestnicy będą mieli obowiązek zapisywać swoje ruchy na wydzielonych do tego kartkach.
\item Każdy stanowisko do gry ma swój numer identyfikacyjny, który obowiązuje przy rozgrywkach.
\item W sali, w której obywa się turniej szachowy zawodnicy jak i widzowie muszą zachować bezwzględną ciszę, aby nie przeszkadzać graczom w rozgrywce.
\item Jeśli jakiś uczestnik turnieju lub widz będzie podpowiadał innemu uczestnikowi, zawodnik otrzymuje od prowadzącego ostrzeżenie, w wypadku powtórzenia się sytuacji gracz któremu pomoc została ponownie udzielona może zostać zdyskwalifikowany z dalszych rozgrywek przez prowadzącego.
\item W turnieju obowiązuje punktacja
\begin{itemize}
	\item Za zwycięstwo - 1pkt
	\item Za remis - 0,5pkt
	\item Za porażkę - 0pkt
	\item Punkty pomocnicze, wykorzystywane gdy kilku zawodników ma taką samą liczbę punktów głównych, przyznawane po zakończeniu etapu (eliminacji lub finału) według schematu:
	\begin{itemize}
		\item Za zwycięstwo - punkty zawodników z którym dany zawodnik wygrał
		\item Za remis - połowę punktów zawodników z którym dany zawodnik zremisował
		\item Za porażkę - Nie są przyznawane punkty pomocnicze
	\end{itemize}
	\end{itemize}
\end{enumerate}
\item W razie rezygnacji lub wykluczenia zawodnika z turnieju, rozgrywki, które zagrał nie zostają anulowane, a osoby, które się z nim spotykają w dalszych rozgrywkach wygrywają walkowerem otrzymują 1pkt za zwycięstwo
\item W Sali zostało wydzielone pięć części:
\begin{enumerate}
\item Pierwsza, w której znajdują się tylko i wyłącznie osoby rozgrywające mecz
\item Druga, w której znajdują się widzowie bądź gracze, którzy obecnie nie rozgrywają żadnego spotkania
\item Trzecia, w której znajdują się stanowiska do gry w szachy poza turniejem
\item Czwarta, w której znajdują się gracze oczekujący na mecz
\item Piąta, w której znajduje się tylko i wyłącznie prowadzący turniej szachowy bądź osoby, które za zezwoleniem mogą znajdować się w tej strefie
\end{enumerate}
\item Zawodnicy, którzy nie grają lub czekają na swoją kolej w obrębie sali lub w niedalekiej odległości od niej w wypadku wezwania do rozgrywki powinni w trybie natychmiastowym zgłosić się do udziału w spotkaniu. W wypadku niestawienia się do rozegrania meczu zawodnik zostaję zdyskwalifikowany.
\item W przypadku, gdy:
\begin{itemize}
\item Zawodnik utrudnia przeprowadzanie rozgrywek może zostać zdyskwalifikowany z turnieju lub wyproszony z sali przez Prowadzącego.
\item Widz utrudnia przeprowadzanie rozgrywek może zostać wyproszony z sali przez Prowadzącego.
\end{itemize}
\item Udział w turnieju szachowym jest równoznaczny z zaakceptowaniem regulaminu
\end{enumerate}
%\section{Diagram sieciowy}
%Diagram sieciowy ukazuje zależności czasowe, węzły (aktywności), krawędzie (zależności czasowe).


\section{Harmonogram}
\subsection{Harmonogram prac poszczególnych członków zespołu}
\textbf{Mirosława Pelc oraz Piotr Jabłoński wspólna praca programistyczna\\
Mirosława Pelc - Odpowiedzialna w głównej mierze za interfejs graficzny\\
Piotr Jabłoński - Programowanie, algorytmy\\}
\begin{tabular}{|p{9cm}|l|p{3cm}|} \hline
Zadanie & Data rozpoczecia & Data zakończenia\\ \hline
Przygotowanie klas odpowiadających za uczestnika, turniej, rozgrywkęzygotowanie klas odpowiadających za uczestnika, turniej, rozgrywkę & 6.10.2015 & 20.10.2015 \\ \hline
Wyszukikawanie możliwych do wykorzystania elementów dostępnych w bibliotekach graficznych dla języka JAVA& 6.10.2015 & 20.10.2015 - zadanie ciągłe wykonywane przez cały czas trwania projektu\\ \hline
Integracja z bazą danych SQLite do przechowywania uczestników& &\\
Integracja z bazą danych SQLite do przechowywania turniejów&20.10.2015&27.10.2015\\
Integracja z bazą danych SQLite do przechowywania wyników pojedynczych rozgrywek&&\\ \hline
tabela - lista uczestników&27.10.2015&3.11.2015\\ \hline
dodawanie nowego uczestnika&27.10.2015&3.11.2015\\ \hline
usuwanie uczestnika&3.11.2015&10.11.2015\\ \hline
edycja uczestnika&3.11.2015&10.11.2015\\ \hline
dodawanie losowego uczestnika&10.11.2015&17.11.2015\\ \hline
symulacja ilości rozgrywek dla danej liczby uczestników, typu turnieju (systemem szwajcarskim / eliminacje grup)&10.11.2015&17.11.2015\\ \hline
podział graczy na grupy wg listy sortowanej po ustalanych przez prowadzącego turniej (dynamicznie w programie) warunkach takich, jak: kategoria zawodnika, wiek, nazwisko, imię lub przydział manualny&17.11.2015&24.11.2015\\ \hline
tworzenie początkowej listy graczy (sortowanie) do turnieju rozgrywanego systemem szwajcarskim (sortowanie po kategorii, wieku, nazwisko, imię)&17.11.2015&24.11.2015\\ \hline
dobieranie zawodników w pary dla systemu kołowego z eliminacjami w grupach - eliminacje
wybór zawodników przechodzących do finałów w rozgrywkach z eliminacjami
dobieranie zawodników w pary dla systemu kołowego z eliminacjami w grupach - finały&24.11.2015&1.12.2015\\ \hline
dobór zawodników w systemie kołowym (4 tyg!)&&\\ \hline 
lista wyników dla turnieju rozgrywanego systemem kołowym z eliminacjami&1.12.2015&8.12.2015\\ \hline
lista wyników dla turnieju rozgrywanego systemem szwajcarskim&&\\ \hline 
zastosowanie programu do prowadzednia kilku turniejów jednocześnie&&\\ \hline
usprawnienia ergonomii interfejsu&&praca ciągła do końca trwania projektu\\ \hline
usprawnienia estetyczne interfejsu&&praca ciągła do końca trwania projektu\\ \hline

\end{tabular}

\begin{tabular}{|p{9cm}|l|p{3cm}|} \hline
Zadanie & Data rozpoczecia & Data zakończenia\\ \hline
Przygotowanie dokumentacji dla projektu&&cały czas trwania projektu\\ \hline
Rozmowa ze zleceniodawcą na temat projektu &&20.10.2015\\ \hline
Wybór systemu w którym przeprowadzany będzie turniej &&27.10.2015\\ \hline
Okreslenie regulaminu turnieju (czas trwania,  system rozgrywego, określenie zasad uczestnictwa w turnieju, powody do dyskwalifikacji) &&3.11.2015\\ \hline
Opis repozytorium GitHub wykorzystywanego do pracy w projekcie &&17.11.2015\\ \hline
Przygotowywanie kolejnych części dokumentacji na podstawie informacji dostarczonych przez pozostałych członków zespołu&&\\ \hline
Testowanie kolejnych wersji  programu, wyszukiwanie błędów sugestie na temat usprawnień - praca ciągła, do końca trwania projektu&&\\ \hline
Konsultację ze zleceniodawcą na temat ewentualnych poprawek, dodawania nowych funkcjonalności wymaganych przez zleceniodawcę.&&\\ \hline 
\end{tabular}

\section{Dokumentacja}
Przygotowanie środowiska do równoległego opracowania dokumentacji projektu i realizacji przydzielonych zadań poszczególnym członkom zespołu projektowego.

\subsection[Edycja plików dokumentacyjnych]{Edycja plików dokumentacyjnych - każdy członek zespoły niezależnie}
Każdy z członków zespołu edytuje swój plik \LaTeX{} (czlonkowie/nrCzlonka/main.tex) i~umieszcza w nim całość analiz i wyników, które pozwoliły mu zrealizować przydzielone zadanie. Wszystkie pliki graficzne, każdy niezależnie umieszcza w swoim katalogu (czlonkowie/nrCzlonka).

Pierwszą linia w pliku (czlonkowie/nrCzlonka/main.tex), zawiera imię i nazwisko opracowującego członka zespołu:
\begin{lstlisting}
\osoba{Jan Iksiński}
\end{lstlisting}

Każde działanie/zadanie należy DOKŁADNIE opisać podając w poleceniu \s!\zadanieprojektowe! cztery obowiązkowe dane:
\begin{itemize}
\item Rodzaj zadania [Przygotowanie przestrzeni do zespołowej pracy]
\item Data rozpoczęcia [2014-11-01]
\item Data zakończenia [2014-11-02]
\item Aktualny status [zaplanowane do realizacji, w trakcie realizacji, zakończone]
\item dokładny opis realizowanego zadania [powinien zawierać opis, rysunki, tabele, kody napisanych programów]
\end{itemize}

Poniżej znajduje się przykładowy listing dla skróconych dwóch zadań:
\begin{lstlisting}
\zadanieprojektowe{Przygotowanie dokumentacji}{2014-11-01}{2014-11-02}{w trakcie do realizacji}

Poniżej opisujemy całe zadanie zgodnie z konwencją poznaną na NI.
Poniżej opisujemy całe zadanie zgodnie z konwencją poznaną na NI.

Poniżej opisujemy całe zadanie zgodnie z konwencją poznaną na NI. 

%następne zadanie
\zadanieprojektowe{Przygotowanie dokumentacji}{2014-11-03}{2014-11-03}{zakończone}
\begin{figure}[H]
\includegraphics[width=\textwidth]{czlonkowie/1/studzienkizDziura.jpg}
\end{figure}
\end{lstlisting}


\subsubsection{Obsługa GitHuba}
Repozytorium wykorzystywane w projekcie to "GitHub" aby zacząć korzystać z tego repozytorium należy najpierw założyć konto w serwisie \href{https://github.com}{https://github.com}
Wybieramy opcję "Sing up" i wypełnamy formularz rejestracyjny
\begin{figure}
	\centering
	\includegraphics {fig/rejestracja}
	\caption{Formularz rejestracyjny repozytorium GitHub}
	\label{fig:rejestracja}
\end{figure}
Nastepnie z menu na górze po prawej stronie wybieramy opcję "New repository"
\begin{figure}
	\centering
	\includegraphics{fig/new_project}
	\caption{Tworzenie nowego repozytorium}
	\label{fig:new_project}
\end{figure}
Uzupełniamy dane dotyczące projektu. Musimy mu nadać nazwę, możemy opcjonalnie dodać opis tworzonego repozytorium, oraz zdecydować czy projekt będzie publiczny czy prywatny

\begin{figure}
	\centering
	\includegraphics{fig/new_project_2}
	\caption{Uzupełniamy dane na temat projektu}
	\label{fig:new_project2}
\end{figure}
Teraz możemy dodać kolejnych uczestników projektu wybierając z menu opcję "New collaborator"
Uczestników możemy wyszukiwać  według róznych kryteriów
\begin{figure}
	\centering
	\includegraphics{fig/collaborator}
	\caption{Dodawanie nowego uczestnika projektu}
	\label{fig:collaborator}
\end{figure}
\begin{figure}
	\centering
	\includegraphics{fig/add_collaborator}
	\caption{Mamy możliwość wyszukiwania nowych członków według różnych kryteriów}
	\label{fig:add_collaborator}
\end{figure}
Aby mieć możliwość wysyłania plików do repozytorium musimy zainstalować program na swoim systemie w tym celu wchodzimy na stronę \href{https://desktop.github.com/}{https://desktop.github.com/}.\\ Program możemy zainstalować w systemach:
\begin{enumerate}
\item Windows 7
\item Windows 8/8.1
\item Windows 10
\end{enumerate}
Starsze wersję systemów operacyjnych nie są wspierane\\
Dostępna jest również wersja dla komputerów MAC z systemem OS X 10.9 lub nowszym
\begin{figure}
	\centering
	\includegraphics{fig/gitdownload}
	\caption{Przycisk umożliwiający pobranie programu}
	\label {fig:gitdownload} 
\end{figure}



