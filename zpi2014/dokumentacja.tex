\pdfinfo{
/ModDate (\pdfcreationdate)                                   
/Producer (pdfLaTeX)                                    
}


\documentclass[fleqn,oneside,openany,a4paper,11pt]{book}

\usepackage{color}
\usepackage[utf8]{inputenc}
\usepackage[breaklinks]{hyperref} %
\usepackage{pracadok}
\usepackage{longtable}
%\usepackage{array}
%\usepackage{geometry}
%\usepackage{fancyhdr}
\usepackage{float}

\pdfcompresslevel=9

\def\uwaga#1{}

\begin{document}
\let\s\lstinline
\lstset{inputencoding=utf8, extendedchars=true,literate={ą}{{\k{a}}}1 {ć}{{\'c}}1 {ę}{{\k{e}}}1 {ł}{{\l{}}}1 {ń}{{\'n}}1 {ó}{{\'o}}1 {ś}{{\'s}}1 {ż}{{\.z}}1 {ź}{{\'z}}1 {Ą}{{\k{A}}}1 {Ć}{{\'C}}1 {Ę}{{\k{E}}}1 {Ł}{{\L{}}}1 {Ń}{{\'N}}1 {Ó}{{\'O}}1 {Ś}{{\'S}}1 {Ż}{{\.Z}}1 {Ź}{{\'Z}}1}

% Tytuł
\def\autor{Informatyka Stosowana III rok}
\def\tytul{\textbf{\LARGE Zespołowe Przedsięwzięcie Inżynierskie}}
\def\promotor{~}
\def\miejscerokwydania{Nowy Sącz \today}
\def\nazwauczelni{PAŃSTWOWA WYŻSZA SZKOŁA ZAWODOWA}
\def\imienia{INSTYTUT  TECHNICZNY}
\def\wydzial{Kierunek Informatyka Stosowana}

\thispagestyle{empty}
{
\hbox{}\vskip 0.3\textheight
\hspace{1cm}
\centering
\vbox{
\noindent\textbf{\Huge Regulamin \\ \vspace{0.3cm}turnieju szachowego}\vspace{0.5cm}\\
}
\definecolor{tlo}{rgb}{.7,.7,.7} 
\lstset{language=bash,commentstyle=\scriptsize,backgroundcolor=\color{tlo},%
basicstyle=\scriptsize}

%spis tresci
{\footnotesize\tableofcontents}

\setcounter{chapter}{0}
%Opis wykonywanego zadania
%Cel
%Zakres prac
%Interesariusze
\chapter{Piotr Jabłoński}

%Mariusz Lorek - kierownik zespołu
\chapter{Piotr Jabłoński}

%Piotr Jabłoński		
\chapter{Piotr Jabłoński}

%Mirosława Pelc
\chapter{Piotr Jabłoński}


%Usuwa numeracje z naglowka. Zapewnia  dodanie do spisu tresci
\setcounter{secnumdepth}{-1}


%Gdy mamy dużą bibliografię to możemy wybierać pozycje,
%które cytujemy
%\nocite{ad-tg-80}

%Dodaje wszystkie pozycje z bibliografii
%\nocite{*}

%Po kazdym dodaniu nowej pozycji bibliograficznej
%z katalogu glownego uruchom: bibtex pracadyp
%\bibliographystyle{pdplain}
%\bibliography{tex/pracadyp}

\begin {thebibliography}{11}
\bibitem{Balcerzak2005} Balcerzak J., Pansiuk J.: \emph{Wprowadzenie do kartografii matematycznej}, Warszawa, OWPW~2005.
\bibitem{Barrett} Barrett R. i inni: \emph{Templates for the Solution of Linear Systems: Building Blocks for Iterative Methods1}, wersja elektorniczna Mathematics http://www.siam.org/books.
\bibitem{bjork} Bjork A., Dahlquist G.: \emph{Numerical Methods in Scientific Computing}, Philadelphia, SIAM~2002.
\bibitem{CCITTG4}CCITT, \emph{Facsimile Coding Schemes and Coding Control Functions for Group 4 Facsimile
Apparatus, Recommendation T.6, Volume VII, Fascicle VII.3, Terminal Equipment and
Protocols for Telematic Services, The International Telegraph and Telephone Consultative Committee (CCITT)}, Geneva, CCITT~1985.
\bibitem{drwal:mathematica2000} Drwal G, i in., \emph{Mathematica 4}, Gliwice, WPKJS~200.
\bibitem{Gdowski1982} Gdowski B.: \emph{Elementy geometrii rózniczkowej w zadaniach}, Warszawa, PWN~1982.
\bibitem{Gotlib2007} Gotlib D., Iwaniak A., Olszewski R.: \emph{GIS obszary zastosowań}, Warszawa, PWN~2007.
\bibitem{INTERGRAPHFileFormat1994} INTERGRAPH: \emph{INTERGRAPH RASTER FILE FORMAT REFERENCE GUIDE}, Alabama, Intergraph Corporation~1994.
\bibitem{Januszewski2006} Januszewski J.: \emph{Systemy satelitarne GPS, Galileo i inne}, Warszawa, PWN~2006.
\bibitem{kielbasinski1992}: Kiełbasiński A., Schwetlick H.: \emph{Numeryczna algebra liniowa}, Warszawa, WNT~1992.
\bibitem{Kincaid2006} Kincaid D.: \emph{Analiza numeryczna}, Warszawa, WNT~2006.
\bibitem{Lamparski2001}Lamparski J.: \emph{Navstar GPS od teorii do praktyki}, Olsztyn, WUW-M~2001.
\bibitem{Levine1994} Levine J.: \emph{Programowanie plików graficznych w C/C++}, New York, Wiley~1994.
\bibitem{Longley2006} Longley P. i inni: \emph{GIS teoria i praktyka}, Warszawa, PWN~2006.
\bibitem{GML:opengis} Open Geospatial Consortium Inc.: \emph{OpenGIS Geography Markup Language (GML) Encoding Standard, Version: 3.2.1},  OGC~2007.
\bibitem{GML:opengisimplemntation} Open Geospatial Consortium Inc.: \emph{OpenGIS® Geography Markup Language (GML) Implementation Specification}, OGC~2004.
\bibitem{Opera2002} Opera J.: \emph{Geometria róniczkowa i jej zastosowania}, Warszawa, PWN~2002.
\bibitem{Poczobut1982Geogeza} Odlanicki-Poczobut M.: \emph{Geodezja}, PPWK~1982.
\bibitem{Li2007}Li Y. i inni: \emph{GML Topology Data Storage Schema Design}, Chiba University~2007.
\bibitem{li2004GMLstorage}Li Y., Li J., Zhou S.: \emph{GML Storage}, A Spatial Database Approach,ER (Workshops), str 55-66, 2004.
\bibitem{Sayood2002} Sayood K.: \emph{Kompresja danych}, Warszawa, Rm~2002.
\bibitem{G52003} \emph{The Technical Instruction G-5, The Ground Cadastre and Buildings, The Main Surveying and
Cartographic Bureau}, Warszawa 2003.



\end {thebibliography}


\listoffigures

%\listoftables %Każdy członek zespołu musi dołożyć minimum 5 pozycji bibliograficzny, które posłużyły mu do opracowania zadanego fragmentu projektu.
\end{document}

To robi tylko kierownik projektu:
Wprowadź polecenie ze swoimi danymi
svn co https://riouxsvn.com/svn/zpi2014 --username aligeza

w bieżącym katalogu zostanie utworzony katalog z nazwą repozytorium, u mnie zpi2014

Skopiuj do niego całą zawartość projektu do katalogu zpi2014
przejdź do niego i wprowadź
svn add * --force
to polecenie dodało wszystkie pliki i podkatalogu do lokalnej kopii repozytorium.


Poniższe polecenie wyśle dane na serwer svn
svn commit -m "inicjacja dokumentacji projektu"


Teraz każdy członek z zespołu wykonuje polecnie ze swoim loginem!!!!!!!
Wprowadź polecenie ze swoimi danymi
svn co https://riouxsvn.com/svn/zpi2014 --username aligeza
Już każdy posiada swoje repozytorium do kompilacji.
Zmiany nanosić w swoim pliku i po porawnej kompilacji
trzeba przesłać na serwer z adekwatnym komentarzem
svn commit -m "Model stanowiska do zobrazowań"