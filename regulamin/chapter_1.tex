\textbf{}
\vbox{
\centering	\noindent\textbf{\Huge Regulamin turnieju szachowego}\vspace{0.5cm}}\\
\begin{enumerate}
\item Obowiązują przepisy gry międzynarodowej federacji szachowej (FIDE).
\item Każdy z zawodników powinien się kierować zasadami fair play.
\item Zasady rozgrywki:
\begin{enumerate}
\item Turniej rozgrywany jest w dwóch fazach: grupowej i finałowej.
\item Ilość grup jest zależna od liczby uczestników i ustala ją prowadzący.
\item W fazie grupowej prowadzone są rozgrywki, wedug zasady "każdy z każdym" w danej grupie
\item W fazie finałowej zawodnicy wyłonieni z grup (liczbę osób wychodzących z grup ustala prowadzący) grają między sobą.
\item Jeżeli zawodnicy grali ze sobą w rundzie eliminacyjnej to w finale przyjmuje się wynik rozgrywki z eliminacji
\item Czas trwania turnieju jest ograniczony, podany przez organizatora
\begin{itemize}
\item \textbf{10 minut} na przyjmowanie zgłoszeń(rejestrację), \textbf{10 minut} na losowanie spotkań, \textbf{10 minut} na rozegranie partii, każdy zawodnik ma 5 minut na wykonanie swoich posunięć.
\end{itemize}
\item Zawodnicy mają do dyspozycji zegar analogowy lub cyfrowy z 2 tarczami lub wyświetlaczami umożliwiający odmierzanie czasu rozgrywki dla każdego z zawodników osobno
\item Za zajecie ustalonych przez prowadzącego miejsc w turnieju zawodnicy otrzymują nagrody przewidziane przez organizatora.
\item Jeśli prowadzący ustali, uczestnicy będą mieli obowiązek zapisywać swoje ruchy na przeznaczonych do tego kartach.
\item Każdy stanowisko do gry ma swój numer identyfikacyjny, który obowiązuje przy rozgrywkach.
\item W sali, w której obywa się turniej szachowy zawodnicy jak i widzowie muszą zachować bezwzględną ciszę, aby nie przeszkadzać graczom w rozgrywce.
\item W turnieju obowiązuje punktacja
\begin{itemize}
	\item Zwycięstwo - \textbf{1pkt}, Remis - \textbf{0,5pkt}, Porażka - \textbf{0pkt}
	\item Punkty pomocnicze, wykorzystywane gdy kilku zawodników ma taką samą liczbę punktów głównych, przyznawane po zakończeniu etapu (eliminacji lub finału) według wzoru: suma puktów zawodników z którymi gracz wygrał + połowa sumy punktów z którymi zremisował
\end{itemize}
\end{enumerate}
\item W razie rezygnacji lub dyskwalifikacji zawodnika z turnieju, rozgrywki, które rozegrał nie zostają anulowane, a osoby, które się z nim spotykają w kolejnych rozgrywkach wygrywają walkowerem (otrzymują 1pkt za zwycięstwo)
\item W Sali zostało wydzielone pięć części:
\begin{enumerate}
\item Pierwsza, w której znajdują się tylko i wyłącznie osoby rozgrywające mecz
\item Druga, w której znajdują się widzowie bądź gracze, którzy obecnie nie rozgrywają żadnego spotkania
\item Trzecia, w której znajdują się stanowiska do gry w szachy poza turniejem
\item Czwarta, w której znajdują się gracze oczekujący na mecz
\item Piąta, w której znajduje się tylko i wyłącznie prowadzący turniej szachowy bądź osoby, które za zezwoleniem mogą znajdować się w tej strefie
\end{enumerate}
\item Zawodnicy, którzy nie grają lub czekają na swoją kolej w obrębie sali lub w niedalekiej odległości od niej w wypadku wezwania do rozgrywki powinni w trybie natychmiastowym zgłosić się do udziału w spotkaniu. W wypadku niestawienia się do rozegrania meczu zawodnik zostaję zdyskwalifikowany.
\item W przypadku, gdy:
\begin{itemize}
\item Zawodnik który korzysta z niedozwolonej pomocy lub utrudnia przeprowadzanie rozgrywek  może zostać zdyskwalifikowany z turnieju lub wyproszony z sali przez Prowadzącego.
\item Widz utrudnia przeprowadzanie rozgrywek może zostać wyproszony z sali przez Prowadzącego.
\end{itemize}
\item Udział w turnieju szachowym jest równoznaczny z zaakceptowaniem regulaminu
\end{enumerate}
